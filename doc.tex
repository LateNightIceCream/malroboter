\documentclass[12pt]{article}

\include{preamble}


\title{Malroboter mit dem Raspberry Pi}
\subtitle{Filmbüro Wismar}
\author{Richard Grünert und Josefine Richey}
\date{12/2020}

\begin{document}

\maketitle

\tableofcontents

\pagebreak

\include{vorbereitung}

\include{durchfuehrung}

\section{Fazit}
Das Schöne an diesem Projekt sind die Möglichkeiten der Abwandlung. Je nach Altersklasse kann man dabei Umfang des Projektes und den Anteil, der durch die KursteilnehmerInnen bearbeitet wird, variieren.\\
Mit jüngeren TeilnehmerInnen kann der Fokus auf den Zusammenbau des Arms und die Programmierung gelegt werden. Die Verkabelung der Motoren und technische Arbeiten sollte hier schon vorbereitet sein. Älteren TeilnehmerInnen kann man die Verkabelung auch selbst zutrauen, und vielleicht bietet sich sogar die gänzlich eigenständige Umsetzung des Projekts inklusive Anlegen der Projektdateien auf dem Raspberry-Pi an. Das Vorbereiten der Materialien empfiehlt sich aber trotzdem.\\
\\
Kein Projekt funktioniert gänzlich fehlerfrei und unproblematisch. Auch wir sind bei der Erstdurchführung und dem zweiten Anlauf über einige Stolpersteine gefallen, die wir gern mit euch teilen wollen.\\
So empfiehlt sich für den Zusammenbau, kräftige und stabile Pappe zu benutzen, um auch einen entsprechend stabilen Roboterarm bauen zu können. Heißkleber ist bei diesem Projekt nahezu unabdingbar, und auch die Verwendung von Zahnstochern macht einiges einfacher. Wir haben im ersten Versuch Schaschlikspieße verwendet, die haben den Arm allerdings unbeweglicher gemacht.\\
Das Originalprojekt wurde mit einem Calliope umgesetzt. Der Calliope 2.0 Mini ist tatsächlich nicht teurer als ein Raspberry-Pi, dafür natürlich nicht so universell einsetzbar.\\
Trotzdem sind wir - angesichts der Komplexität des Projektes - der Meinung, dass ein Calliope hier einiges einfacher gestalten kann. \href{https://calliope.cc/projekte}{Auf der offiziellen Website}\footnote[4]{https://calliope.cc/projekte} gibt es zudem viele weitere Projekte, die damit umgesetzt werden können.\\
Nichtdestotrotz: Mit einem Raspberry-Pi ist es ebenso machbar, wenn auch ein bisschen kniffeliger.\\
Empfehlen würden wir das Projekt für TeilnehmerInnen ab ewa 10Jahren und einen Zeitrahmen von etwa zwei bis drei Stunden.\\

\include{anhang}

\end{document}
